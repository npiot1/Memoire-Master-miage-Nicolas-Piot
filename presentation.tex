

\chapter{Présentation}
\label{chap:presentation}
%\minitoc

\section{Introduction}
Il est certain que l'informatique prend une place de plus en plus importante pour tout le monde. Que ce soit pour les utilisateurs ou les concepteurs, le monde informatique se complexifie et s'élargie en différents domaines si bien qu'aujourd'hui il parait nécessaire d'avoir certaines notions, même basiques, pour mieux appréhender le monde qui nous entoure. Si nous prenons le cas français, l'informatique est enseignée pour l'instant seulement en terminale S dans une spécialité que l'élève choisit parmi d'autres spécialités (en l'occurrence la spécialité ISN : informatique et science du numérique). En mathématique, nous voyons aussi quelques notions algorithmiques basiques mais qui ne sont pas accessibles à toutes les filières. 

Toujours en France, l’Éducation nationale a annoncé la création du Capes d’informatique en 2020. \cite{3} Une décision qui consacre l’enseignement des sciences du numérique en tant que nouvelle discipline scolaire. Cette enseignement prendra effet dans un premier temps dans les lycées dans le but de sensibiliser les élèves aux concepts informatiques basiques. L'objectif n'est pas de former des ingénieurs informatiques mais de donner des notions pour que les élèves se donnent une idée concrète de l'informatique et voir si cela leur plaît.

Un nouveau problème se pose, comment aborder le sujet de l'informatique de façon ludique à des personnes qui ne sont à la base pas forcément intéressées par ce sujet ? Quelles solutions d'apprentissages peut-on mettre en place pour familiariser les enfants et adolescents aux concepts informatiques pour qu'ils comprennent mieux le monde qui nous entoure et surtout le monde de demain ? C'est ici dans ce mémoire que nous allons tenter de répondre à cette question en abordant différents concepts, en parlant premièrement de l'existant puis ensuite de solutions proposées. Nous nous attarderons sur un point vue enfance et adolescence pour englober différentes situations possibles.

C'est pour cela que dans un premier temps nous allons parler des concepts qui me semblent importants en informatique, en partant d'un cas plutôt général vers des cas plus particuliers. Nous allons aussi parler des différentes contraintes liées à la problématique. Ensuite, nous aborderons l'existant dans l'état de l'art pour le comparer et faire le lien avec mes propositions dans la partie suivante. Ces propositions seront analysées (forces et faiblesses), puis nous finirons avec les évolutions possibles et une synthèse.

\clearpage


\section{Étude et contexte du problème}
\subsection{Les concepts informatiques}
Les domaines informatiques sont nombreux. On peut penser dans un premier temps au développement et à la programmation, à la modélisation et à la conception de systèmes d'informations. Il y a également l'aspect base de données, réseaux, l'algorithmique, les commandes Unix etc... Chacun de ces domaines ont leurs spécificités et sont plus ou moins accessibles à l'apprentissage selon les âges. Il parait par exemple difficile d'apprendre à des enfants d'administrer une base de données mais nous pouvons leur inculquer des manières de réfléchir en s'écartant des conventions et surtout de façon ludique. Ce qui est important n'est pas connaître les détails d'un domaine informatique mais en appréhender les bases dans un environnement non complexe pour l'enfant ou l'adolescent. Si, de surcroît, cet environnement est familier, cela facilitera grandement les choses. Par conséquent, il semble important de trouver pour chacun de ces domaines des solutions intelligentes pour l'apprentissage et la familiarisation avec la technologie.

\subsection{Acessible aux enfants ?}
Ce qui peut être clair pour nous ne l'est pas forcément pour les autres. Cette phrase est importante quand on travaille dans l'apprentissage. Avec un sujet comme celui-ci, la première question que l'on se pose est : est-ce que apprendre des domaines informatiques à des enfants est une idée viable ? Est-ce qu'un enfant peut être sensible à un apprentissage de ces matières qui semblent à première vue compliquées ? En me fiant aux travaux de chercheurs de l'université de Canterbury en Nouvelle Zélande \cite{1} on peut déjà avoir une idée de la sensibilité des enfants à l'apprentissage de l'informatique. En l'occurrence, la première étude indique que l'apprentissage dépend fortement des outils employés et du contexte. Il est notamment important de laisser les enfants découvrir les concepts par eux même et il faut leur offrir des opportunités pour leur permettre de continuer leur apprentissage. Les chercheurs en ont défini alors des critères majoritaires permettant une meilleure introduction à la programmation et une meilleure réceptivité de l'enfant. Ici, on ne parle pas forcément de moyen pour apprendre mais plus du contexte d'apprentissage qui permet d'évoluer dans un domaine de façon efficace. Alors oui, apprendre l'informatique à des enfants c'est possible mais avant de vouloir familiariser ou sensibiliser il faut bien s'y prendre. Ce que l'on sait des résultats de cette étude et également confirmé par d'autres \cite{8} c'est qu'enseigner l'informatique aux enfants est non seulement possible mais également recommandé. Les enfants de 4 et 5 ans peuvent apprendre les fondements informatiques avant même de savoir lire et écrire. Finalement les chercheurs ont défini différents niveau d'apprentissage, de 5 à 10 ans avec peu d'abstraction il peuvent apprendre les rudiments des fonctions, variables, itérations, les structures indexés ainsi que les conditions.
Les environnements les plus appropriés pour ce genre d'apprentissage sont les "drag and drop" ou "glisser-"déposer" (Voir annexe). Avec le temps, il est possible d'apprendre entièrement ces concepts pour des enfants de 14 ans et plus.

\newpage

\subsection{Enseigner l'informatique en primaire ?}

L'informatique pour l'instant ne s'apprend pas à l'école primaire, on peut alors penser que ce n'est pas impactant de faire cela. Plus encore, l'écriture, les mathématiques ou encore l'apprentissage de la langue semblent être plus fondamentale. Les élèves ont déjà parfois des programmes assez chargés, il serait donc non convenu de leur rajouter des choses. De plus, on observe que certains élèves de primaire possèdent des lacunes en lecture et en écriture en rentrant au collège et que ainsi il est plus prioritaire de pallier ces dites lacunes plutôt que de rajouter des enseignements. \cite{4} \cite{5}

Dans le contexte de la problématique de ce mémoire, l'objectif n'est pas forcément d'instaurer des matières foncièrement informatiques à l'école primaire (même si ceci peut être par contre intéressant au lycée), mais plutôt de trouver un moyen d'introduire des concepts avec d'autres matières qui permettent une première sensibilisation de l'élève même sans savoir que ce dernier fait quelque chose qui a un rapport avec l'informatique. En connaissance de ceci, la mission est donc de se concentrer sur des moyens efficaces (soit déjà existants soit avec ma propre contribution) de familiarisation de l'informatique. Cependant, on peut aussi réfléchir au contexte d'implantation dans un milieu éducatif même en primaire.

\subsection{L'intérêt ?}

La question de l'intérêt d'apprendre ou de familiariser des enfants à l'informatique peut aussi se poser. Comme abordé légèrement dans l'introduction, apprendre des notions informatiques aux enfants leur permet de se préparer dans le monde dans lequel nous vivons aujourd'hui. Communication, éducation, banque, réseaux sociaux, sécurité, shopping, domotique... L'informatique est présente dans énormément d'aspects de note vie. 
L'apprentissage de ce domaine permet aussi de posséder de nouvelles façons de réfléchir. Algorithme, récursion, heuristique, même sans connaître ces mots les enfants peuvent appréhender les bases de ces concepts. On peut également dire que l'informatique peut être un facteur de créativité chez l'enfant liant la résolution de problèmes (jeux, animation, puzzle ...) et l'amusement. Savoir comment certaines choses fonctionnent peut aussi être intéressant (on peut par exemple penser à un jeu vidéo). Toutes ces choses font que l'intérêt est non négligeable et qu'il sera de plus en plus important avec le temps. \cite{9}

\subsection{En résumé}

L'informatique est donc un domaine à part ? La question est légitime, comme dit précédemment, cela semble être plus simple d'apprendre des additions à un enfant que de lui apprendre l'algorithmique. Pourtant, si on réfléchit bien au problème il est possible de créer des structures adaptées aux enfants en prenant par exemple un environnement ludique. L'important dans l'apprentissage va être le contexte, si l'on donne aux enfants les outils nécessaires à l'investissement dans l'apprentissage de l'informatique alors, cet apprentissage peut être semblable et surtout accessible en comparaison avec n'importe quelle autre matière. L'important est que ce soit parlant pour l'élève. On peut notamment penser à des expériences qui ont démontrées que si l'on présente un exercice comme un exercice de géométrie ou de dessins, le taux de réussite est plus élevé dans le premier cas que dans le second. \cite{6} \cite{7}

Par conséquent, si nous présentons l'informatique de façon ludique et non complexe nous pourrons avoir des résultats sûrement similaires. Mais le problème étant toujours présent, comment faire  exactement ?

%%% Local Variables: 
%%% mode: latex
%%% TeX-master: "isae-report-template"
%%% End: 
