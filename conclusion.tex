\chapter{Conclusion}

L'informatique est à la portée de tous. Dès l'enfance des outils sont à notre disposition pour apprendre des fondements informatiques qui peuvent paraître pourtant aux premiers abords compliqués. Ces outils sont en plus très nombreux et ce depuis les années 60. On reste cependant déçu par le fait qu'avec toutes ces ressources il n'y a pas clairement de dispositifs éducatifs créés par l'éducation nationale en France, même si cela est en bonne voie pour 2019. On peut noter qu'il y tout de même une volonté d'introduction à à l'algorithmique avec Scratch. \cite{25}

Avoir des notions informatiques est essentiel aujourd'hui pour comprendre le monde. Agriculture, industrie, médecine, divertissement, commerce ... L'informatique est absolument présente partout. Pourtant, le constat que l'on fait est que malgré cette présence quasi exclusive au final la majorité des gens ne savent pas écrire une ligne de code. On ne dit pas que tout le monde doit savoir coder, mais il est tout de même important d'avoir un aperçu de cette discipline, même général.

Sensibiliser les enfants permet également d'appréhender ce domaine sans posséder de stéréotype. Il est facilement visualisable que l'informatique est un domaine où il y a une majorité d'hommes. Pourtant, sur certaines études par exemple sur Scratch, nous avons vu que les filles se débrouillaient en moyenne mieux que les garçons sur certains aspects de la programmation. Pourtant, ces dernières, même si elles en ont les capacités, sont moins présentes en informatique à l'université. Appréhender l'informatique dès le jeune âge permet de casser ces stéréotypes et voir directement si ce domaine nous plaît ou non.

D'un point de vue purement technique, il a aussi été démontré que être sensibilisé à l'informatique permet de développer des nouvelles façons de résonner et de résoudre des problèmes de telle façon que cela n'existe pas encore aujourd'hui dans l'éducation. On a même pu voir que des concepts comme la récursion ou la concurrence était abordable même pour les enfants. Ce qui est clairement une preuve que l'informatique est accessible aux enfants. C'est même accessible de façon ludique.

Pourtant à l'école l'informatique en France se résume pour l'instant à l'obtention du B2i (Brevet informatique et internet) qui est pour résumer la preuve que l'élève sait lire des caractéristiques d'un fichier sur un système en particulier (Windows). Or, la programmation par exemple est un langage qui est adaptable sur tout système d'exploitation et même sur les objets connectés, les puces électroniques etc...

Aussi, en se concentrant sur l'état de l'art et l'apport de ce mémoire, il apparaît que l'introduction de l'informatique à l'école (que ce soit au primaire, collège ou bien encore au lycée) est clairement envisageable. C'est d'ailleurs un projet de l'éducation nationale pour le lycée puisqu'en 2021 entrera en vigueur le nouveau bac et que dès la rentrée 2019 il y aura un enseignement commun SNT (Sciences numériques et technologie) en seconde général qui parlera d'internet et du web, d'objets connectés de photos numériques etc... \cite{69}

En conclusion, familiariser les concepts informatiques aux enfants est d'abord possible et ensuite important pour appréhender le monde de demain. Cela permet de développer de nouvelles compétences et de casser, dès le plus jeune âge, les stéréotypes sur ce domaine pour permettre qu'il soit accessible à tous.

"\textit{Everybody in this country should learn to program a computer because it teaches you how to think}"

-Steve Jobs

